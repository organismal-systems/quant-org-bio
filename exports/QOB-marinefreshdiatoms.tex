\section{📖️ Why are diatoms larger in marine than freshwater habitats?}

\href{https://en.wikipedia.org/wiki/Allometry}{Allometry} is the study of how organismal traits vary with organism size.
Among aquatic\footnote{In Organismal Biology, the term \textit{aquatic} is usually interpreted as referring to a water habitat that may be either marine (saltwater) or freshwater.} organisms, size is associated with variations in numerous other traits and processes, across all of the organizational scales relevant to Organismal Biology.
Examples include physiological traits such as respiration and nutrient uptake rates, organism-level traits such as intrinsic rate of growth and fecundity, behavioral traits such as locomotion and sensing, population-level traits such as reproduction and mortality rates, and ecological traits such as trophic interactions and carbon sequestration.
\href{/allometry}{Allometry} is a useful tool for understanding linkages between these and many other characteristics of aquatic organisms.

\href{https://en.wikipedia.org/wiki/Diatom}{Diatoms} are a large group of single-celled algae inhabiting aquatic environments world-wide.
Diatoms are diverse, with about 12,000 known species\footnote{Estimates of the number of species ranging from 20,000 to as high as 200,000.}.
Across this diversity, diatoms encompass a wide range of sizes ($2 -2000 \mu m$), and morphologies with numerous \href{https://diatomimaging.com/gallery1/}{beautiful and interesting features}.
Many of these features, such as pores and spines, likely serve functional roles in nutrient acquisition, predator defence and depth regulation.

Diatoms are abundant, and are estimated to produce 20-50\% of new oxygen generated worldwide each year.
Diatoms are major contributors to primary production, are important consumers of dissolved nutrients and have strong trophic interactions with zooplankton grazers, heterotrophic protists and aquatic pathogens.

\begin{figure}[!htbp]
\centering
\includegraphics[width=0.625\linewidth]{files/007a69803b8bc933ed0c6d95198e9803.png}
\caption[]{Images from \href{https://en.wikipedia.org/wiki/Frustule}{Wikipedia} of frustules from four species of diatoms (scale bar = 10 micrometres in a, c and d and 20 micrometres in b).}
\label{frustule}
\end{figure}

Diatoms are encased in hard silica shells called \href{https://en.wikipedia.org/wiki/Frustule}{frustules} (Figure~\ref{frustule}).
Due in part to the weigh of these frustules, diatom cells can sink relatively rapidly.
This makes diatoms significant contributors to sequestration into deep water of carbon fixed by photosynthesis in surface layers.

In 2009, \cite{Litchman_2009} published a survey of diatom size distributions across a number of marine and freshwater habitats.
They found that, though there is a large diversity of sizes in all habitats, the diatom size distribution in marine habitats is consistently larger than the size distribution in freshwater habitats.
Litchman \textit{et al.} asked,

\begin{quote}
Why does a disparity in size distributions exist between marine and freshwater habitats?
\end{quote}

Litchman \textit{et al.} noted that diatoms follow the usual allometries for nutrient uptake and growth:
Smaller cells have, relative to metabolic demands, faster uptake of nutrients such as nitrogen and phosphorus. Smaller cells also have better ability to absorb very dilute nutrients.
Furthermore, smaller cells have faster reproductive rates when replete with nutrients than larger cells.
Each of these factors favor smaller cells, raising an even more fundamental question,

\begin{quote}
Why do large diatoms exist, when small ones have advantages in both nutrient uptake and maximum growth rate?
\end{quote}

A hypothezed factor frequently cited as favoring large cells is size-specific predation: the idea that smaller cells have a higher risk of predation, or of infection by pathogens, than larger cells.
Larger cells would then be maintained not because they can grow faster but because they suffer lower predator- or pathogen-induced mortality rates than small cells.
This type of regulation of a population by higher trophic levels is called \href{https://en.wikipedia.org/wiki/Population\_ecology\#Top-down\_and\_bottom-up\_controls}{top-down control}.

Litchman \textit{et al.} hypothesized that additional answers to both of these questions are found in \href{https://en.wikipedia.org/wiki/Population\_ecology\#Top-down\_and\_bottom-up\_controls}{bottom-up control}, modulated by the phenomena in oceans and lakes called \href{/mixedlayers}{Mixed layers}.