\section{🧮 Effects of flow on molecular transport mechanisms}

A very useful set of formulas approximating $\mathcal{Sh}$ relevant to many environmental particles is given by Kiorboe et al. (2001). Here is a plot of $\mathcal{Sh}$ as a function of $\mathcal{Re}$ and $\mathcal{Sc}$ for spherical particles:

In this plot, the cyan, red and blue lines represent the Sherwood number, $\mathcal{Sh}$, for different Schmidt numbers, $\mathcal{Sc}$.
$\mathcal{Sc}$is the ratio of the fluid kinematic viscosity to the solute's diffusion coefficient.
That is, higher $\mathcal{Sc}$ implies more viscous fluids or less diffusive solutes.
Conversely, lower $\mathcal{Sc}$ implies less viscous fluids or more diffusive solutes.

The dashed black line represents $\mathcal{Sh}=1$; that is, the case when flow has no effect on mass flux.

\begin{figure}[!htbp]
\centering
\includegraphics[width=0.7\linewidth]{files/RS3_2-efd9df72d48eed69b6a28522a69d3556.png}
\end{figure}

\subsection{Sherwood number calculator}

As you can see in the plot above, under some conditions mass transport is greatly increased by fluid flow.
But, under other conditions, the increase is minimal.
Which applies to a given situation?

The text panel below enables you to enter the particle size, density and surface solute concentration, and the fluid viscosity, density and solute concentration.
The worksheet will then calculate for you the corresponding $\mathcal{Re}$, $\mathcal{Sh}$ and mass transport rates.

In many environmental problems, important issues involve the amount of mass transport per unit mass of particle.
For example, a pollution problem may involve knowing whether a given mass of pollutant divided into many small particles has a different environmental effect than the same mass in a few larger particles.
In the example of nutrient uptake by phytoplankton, an ecological question might revolve around whether a given biomass of small cells is more effective at competing for nutrients than an equal biomass of large cells.
The calculator enables you to address those questions by calculating the mass transport per unit mass of particle.

\begin{figure}[!htbp]
\centering
\includegraphics[width=0.7\linewidth]{files/RS3_1-ecfbbb20731816e5af7e77e554ca59f9.png}
\end{figure}