\section{🏞️ What are the forces acting on immersed particles?}

For most particles relevant to Organismal Biology, three types of forces most strongly affect movement\footnote{In a few rare but interesting cases, other forces such as magnetism are important!}:

\begin{enumerate}
\item \textbf{Gravity forces}: Every particle has mass, \textit{M}.

The force \href{https://en.wikipedia.org/wiki/gravity}{gravity} exerts on a particle is
\begin{equation}
F_{gravity} = g \times M
\end{equation}
where $g = 9.81 \frac{m}{s^2}$ is the gravitational acceleration\footnote{Note that $F_{gravity}$ is a \textbf{vector}, which means simply that it has both a \textit{magnitude} ($9.81 \times M$) and a \textit{direction} (down).}.


\item \textbf{Pressure forces}:

As you know from your ears, if you have dived to the bottom of a swimming pool, \href{https://en.wikipedia.org/wiki/pressure}{pressure} increases with depth\footnote{You may also have felt the opposite: ``popping'' in your ears from the decrease in air pressure as an airplane takes off.}.
That implies that the pressure acting on the bottom surface of a particle is stronger than the pressure acting on the top of that particle (because the bottom is deeper than the top).

Since the pressure pushing upwards on the bottom of the particle is stronger that the pressure pushing downwards on the top of the particle, the net force of this pressure is an upward force, which we call \textbf{buoyancy}.
It turns out that this force is always equal to the gravitational force on the mass of fluid that the particle displaces, but opposite in direction (that is, the buoyancy tends to push particles upwards while gravity pulls them downwards).

This gives us the familiar sinking \textit{vs.} floating phenomena: If a particle is denser than the fluid, its mass is greater than the mass of the displaced fluid, and hence it has a net downward force (gravity exceeds buoyancy). Therefore it sinks.
If the particle is less dense than the fluid, buoyancy exceeds gravity, and it floats.

If the particle is moving,


\item \textbf{Viscous forces}:

\href{https://en.wikipedia.org/wiki/viscosity}{Viscosity} is a measure of how ``thick'' a fluid is.
In engineering terms, viscosity is the resistance to the motion of two parallel plates sliding past one another (this motion is called \textbf{shear}).
For example, fluids like honey, molasses and glycerine have much higher viscosity than fluids like water and air.

This means that dragging a particle through a more viscous fluid like honey at a given velocity requires much more force than dragging the same particle at the same velocity through a less viscous fluid like water or air.
\end{enumerate}

Gravity, pressure (including buoyancy) and viscous forces all contribute to the sum of forces on the \textit{left hand side} of the equation for Newton's 2nd Law, as applied to an immersed particle.