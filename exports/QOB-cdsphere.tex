\section{🧮 The Coefficient of Drag}

As described in the \href{rslookup.md}{previous page}, the resistance force $F$ on an immersed organism moving in a fluid is commonly expressed in a ratio known as the Coefficient of Drag:
\begin{equation}
C_d  = \frac{F}{\frac{1}{2} \rho L^2 U^2}
\end{equation}
The Coefficient of Drag is the formula used to calculate the drag forces on an organism from an observation made on a dynamically similar scale model (that is, a model that is geometrically similar and has the same Reynolds number, $\mathcal{Re}$).

The power of this approach is illustrated by the plot below, which shows the drag force on \textit{any} size spherical organism moving at \textit{any} velocity in fluid of \textit{any} viscosity and density.

\begin{figure}[!htbp]
\centering
\includegraphics[width=0.7\linewidth]{files/CdSphere_1-b96e4b26c491f96137843a3436d9456a.png}
\end{figure}

In this plot, note that the scales are logarithmic base 10 for both axes.
Note also that there are several types of data used to define the curve:

\begin{itemize}
\item cyan line (and purple dots) for experimental observations
\item blue and red lines for analytical model calculations.
\end{itemize}

The dash-dot cyan line interpolates between these data in a sensible way, providing a consistent function across many orders of magnitude of Reynolds number.
The interpolation shown by the cyan line is the form in which $C_d$ would typically be calculated in a biomechanical investigation.

\subsection{How to calculate drag forces on organisms}

To obtain the drag force on a spherical organism, use the following steps:

\begin{enumerate}
\item Use the organism's size ($L$) and velocity ($U$), and the fluid's viscosity ($\mu$) and density ($\rho$) to calculate its Reynolds number,

\begin{equation}
\mathcal{Re} = \frac{\rho U L}{\mu}
\end{equation}


\item Take the logarithm, base 10, of the Reynolds number
\item Find the position on the horizontal axis corresponding to that $\log_{10}\mathcal{Re}$
\item Move vertically to the cyan curve; this is the $\log_{10}C_d$.
\item Calculate the coefficient of drag, $C_d = 10^{\log_{10}C_d}$
\item Calculate the drag force,

\begin{equation}
F_{drag} = \frac{1}{2} \rho L^2 U^2
\end{equation}
\end{enumerate}

Some tools to make these calculations easier and more precise are provided in \href{/rs1}{this worksheet}.

\subsection{Trends in the $C_d$ curve}

Several features of the $C_d$ curve are worth pointing out.

\begin{itemize}
\item The denominator in this expression is very similar in form and rationale to our estimate of the \href{./rscharact2.md}{wake momentum}.
The rationale is that, in cases where the drag forces are dominated by increases in wake momentum, the numerator will change in a roughly similar way as the denominator.
By the logic in the \href{./rscscaling.md}{scaling of fluid forces}, wake momentum is dominant when $\mathcal{Re} \gg 1$.
Therefore, we predict that $C_d$ will change relatively slowly at high Reynolds number.
\item In contrast,  at low Reynolds number, we expect the drag from vicous forces to be much higher than from wake momentum.
Therefore, we predict that $C_d$ will be large and (increase with decreasing Reynolds number) when $\mathcal{Re} \ll 1$.
\item In the range $10^4 < \mathcal{Re} < 10^6$, the coefficient of drag has some surprising ``wiggles''. This is due to the onset of \href{https://en.wikipedia.org/wiki/Turbulence}{turbulence}.
\end{itemize}